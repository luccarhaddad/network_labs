\documentclass[brazil, 12pt]{article}

\usepackage[portuguese]{babel}
\usepackage[utf8]{inputenc}
\usepackage[T1]{fontenc}
\usepackage[dvips]{graphicx}
\usepackage{caption}
\usepackage{subcaption}
\usepackage[scale=0.8]{geometry} % Reduce document margins
\usepackage{minted}    
\usepackage{fancyvrb,newverbs,xcolor}
\usepackage{titlesec}
\usepackage{multirow}
\usepackage{indentfirst}
\titleformat*{\section}{\normalsize\bfseries}
\titleformat*{\subsection}{\normalsize\bfseries}
% \usepackage{hyperref}

\begin{document}

%-----------------------------------------------------------------------------------------------
%       CABEÇALHO
%-----------------------------------------------------------------------------------------------
\begin{center}
\textbf{Instituto Tecnológico de Aeronáutica - ITA} \\
\textbf{Redes de Computadores e Internet - CES-35} \\
\textbf{Alunos}: Lucca Haddad e Gustavo Silveira
\end{center}

\begin{center}
\textbf{Laboratório 2 - Construção de Servidor em C}
\end{center}
%-----------------------------------------------------------------------------------------------
\vspace*{0.5cm}

%-----------------------------------------------------------------------------------------------
%       RELATÓRIO
%-----------------------------------------------------------------------------------------------
\section{Implementação}

	A construção do solução cliente-servidor foi realizada utilizando a linguagem C, com a criação de arquivos adicionais \textbf{functions.cpp} na pasta \textbf{utils} para declaração de funções recorrentemente utilizadas tanto pelo servidor quanto pelo cliente. Além disso, há também um \textbf{Makefile} com comandos para compilação e execução dos arquivos definidos no \textbf{README.md}, e dois arquivos \textbf{.txt} para teste de funcionamento da requisição.

	\subsection{Client (Desenvolvido por Gustavo Silveira)}
	
		O código do cliente consiste essencialmente em duas funções: \textbf{receive\_response} e \tetxtbf{main}. A função de recepção de resposta toma como argumentos o socket do servidor e o comando enviado pelo cliente. A partir disso, o código trata as mensagens recebidas do servidor em blocos, realizando validações sucessivas e, no momento de confirmação de bom funcionamento, escreve o resultado da requisição no \textbf{stdout}.
		
		No arquivo \textbf{main}, primeiramente ignora-se o \textbf{SIGPIPE} para evitar quebra do código caso algum dos lados tente escrever em um socket fechado. Em seguida, são realizadas as seguintes operações:
		
		\begin{itemize}
			\item Criação de um socket TCP;
			\item Configuração do tipo do socket e da porta de conexão;
			\item Conversão do endereço IP para a formatação binária;
			\item Conexão com o servidor;
			\item Tratamento e envio de requisições ao servidor.
		\end{itemize}
	
	\subsection{Server (Desenvolvido por Lucca Haddad)}
	
		O código do servidor é consideravelmente mais complexo do que o código do cliente, visto que ele é responsável pelo gerenciamento de requisições de diversos clientes diferentes de maneira simultânea.
		
		Inicialmente, criou-se uma \textbf{struct ClientInfo} para armazenamento do endereço IP do cliente, sua porta específica, o timestamp de último acesso e um ponteiro para o próximo cliente, de forma que seja criada uma lista encadeada com as informações dos clientes conectados.
		
		As seguintes funções auxiliares foram criadas para execução do gerenciamento e recepção/envio de informações dos clientes:
		
		\begin{itemize}
			\item \textbf{update\_client\_access}: Trava o mutex para acesso exclusivo da thread atual e procura pela existência do cliente na lista ligada. Se existir, atualiza seu momento de último acesso e destrava o mutex; caso contrário, cria um novo cliente na lista encadeada e atualiza suas informações.
			\item \textbf{get\_client\_last\_access}: Trava o mutex para acesso exclusivo da thread atual e procura pela existência do cliente na lista ligada. Se existir, devolve o timestamp do último acesso; caso contrário, devolve um número arbitrário que indica inexistência de requisições prévias.
			\item \textbf{handle\_client}: Armazena as informações de IP e de porta do cliente. Em seguida, começa um loop infinito de recepção e de tratamento de requisições. A cada requisição recebida, há o tratamento de erros e a subdivisão em \textbf{MyGet}, \textbf{MyLastAccess} e \textbf{exit}, além da mensagem de erro para mensagens inválidas.
		\end{itemize}
		
		Na função \textbf{main}, assim como no client, ignora-se o sinal do pipe para evitar quebra do código e inicializa-se a estrutura do servidor e o mutex da lista encadeada. Em seguida, são realizadas as seguintes operações:
		
		\begin{itemize}
			\item Criação de um socket TCP;
			\item Configuração do tipo do socket e da porta de conexão;
			\item Bind do socket do servidor às informações especificadas anteriormente;
			\item Escuta das conexões enviadas pelos clientes;
			\item 
		\end{itemize}
	
\section{Testes}

	\subsection{Robustez e Consistência}
	
	
	
	\subsection{Acessos múltiplos}
	
	
	

\end{document}